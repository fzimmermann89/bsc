\chapter{Fourier Transformation}
Die in dieser Arbeit verwendete Form der Fouriertransformation lautet
\begin{equation}
	\mathscr{F} [f(\vec{x})] (\vec{q})
	=
	\frac{1}{(2\pi)^\frac{n}{2}}
	\int_{-\infty}^{\infty}
	f(\vec{x})
	e^{-i\vec{q} \cdot \vec{x} } 
	\dif  \vec{x}
\end{equation}
mit der inversen Fouriertransformation
\begin{equation}
	\mathscr{F}^-1 [\tilde{f}(\vec{q})] (\vec{x})
	=
	\frac{1}{(2\pi)^\frac{n}{2}}
	\int_{-\infty}^{\infty}
	\tilde{f}(\vec{q})
	e^{i\vec{q} \cdot \vec{x} } 
	\dif  \vec{q} \, .
\end{equation}
Für diese gilt unter anderem: 
\paragraph{Faltungstheorem}
\begin{align*}
	\mathscr{F} [f\ast g] & =(2\pi)^\frac{n}{2}\mathscr{F}[f] \mathscr{F}[g]     \\
	\mathscr{F}[fg]       & =(2\pi)^\frac{n}{2}\mathscr{F}[f]\ast \mathscr{F}[g] \numberthis
	\label{eq:faltung}
\end{align*}
Eine Faltung im Realraum entspricht einer Multiplikation im Fourierraum.
\paragraph{Korrelationstheorem}
\begin{align}
\mathscr{F} [f\otimes g] & =(2\pi)^\frac{n}{2}\mathscr{F}[f] \mathscr{F}[g]^*     \\
\label{eq:korrelation}
\end{align}
Eine Korrelation im Realraum entspricht im Fourierraum einer Multiplikation mit dem komplex konjugierten.