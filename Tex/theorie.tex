\chapter{Theoretischer Hintergrund}
\label{c_theorie}
 
\section{Skalare Beugungstheorie}
\section{Fraunhofer-Näherung}
\section{Angular-Spectrum Propagation}
\section{Born-Näherung}


\begin{comment}

\subsubsection{Subsubsection}

\paragraph{Paragraph}



\section{Abbildungen}
Eine Beispiel-Abbildung ist in Abb. \ref{Abb:BspAbbildung} gezeigt.

\begin{figure}
\centering
\includegraphics[width=0.9\textwidth]{images/SchemaErzeugungNachweis.jpg}
\caption[Abbildungstext im Abbildungsverzeichnis]{Abbildungsunterschrift. Abbildung nach \cite{B-SViel}.}
\label{Abb:BspAbbildung}
\end{figure} 



\section{Tabellen}
Eine Vorlage ist in Tabelle \ref{Tab:BspTabelle} gegeben.
 
\begin{table}
\centering
\begin{tabular}{SccrrS}
\hline\hline
{Druck [\si{\milli\bar}]} & Gas &  $T_0$ [K]& $\Gamma^\ast$ & $\langle N \rangle$ & {$R$ [nm]}\\
\hline
1000 & Ar &  300 & 2.325 & 240 & 1,3\\
1000 & Ar &  300 & 2.971 & 426 & 1,6\\
1000 & Xe &  300 & 7.845 & 4.177 & 3,9\\
1000 & Xe &  300 & 10.024 & 6.337 & 4,4\\
\hline
5000 & Ar &  300 & 11.625 & 8.274 & 4,2\\
5000 & Ar &  300 & 14.854 & 12.862 & 4,9\\
5000 & Xe &  300 & 39.226 & 73.863 & 10,1\\
5000 & Xe &  300 & 50.121 & 114.826 & 11,7\\
\hline
10000 & Ar & 300 & 23.250 & 28.812 & 6,3\\
10000 & Ar & 300 & 29.708 & 44.790 & 7,4\\
10000 & Xe & 300 & 78.452 & 257.207 & 15,3\\
10000 & Xe & 300 & 100.243 & 399.848 & 17,7\\
\hline
8000 & Xe &  220 & 127.589 & 617.258 & 20,4\\
8000 & Xe &  220 & 163.029 & 959.574 & 23,7\\
\hline\hline
\end{tabular}
\caption[Text für Tabellenverzeichnis]{Tabellenunterschrift.}
\label{Tab:BspTabelle}
\end{table}

\section{Anführungszeichen}
Anführungszeichen werden global über das Paket \emph{csquotes} gesetzt und wie folgt eingebunden:

\verb+\enquote{Text in Anführungszeichen}+

Ausgabe: \enquote{Text in Anführungszeichen}.



\section{Setzen von Einheiten}
Zum Setzen von Einheiten wird das Package \emph{siunitx} verwendet:

Zahlen: \num{83567}, \num{,23e-16}, \num{3,89+-0,03}

Winkel: \ang{12.3}, \ang{1;2;3}

Wert mit Einheit: $v_{max}=\SI{260}{\m/\s}$, $v_\infty=\SI{173}{\metre\per\second}$, $R = \SI[per-mode=symbol]{8,3144621}{\joule\per\mol\per\kelvin}$

Wert mit Fehler und Einheit: $T_0=\SI{5,234(15)}{\kelvin}$

Celsius: $T_S=\SI{38.6}{\celsius}$ 

Minuten: $t_c=\SI{90}{\min}$

Druck: $p_0=\SI{80}{\bar}$, Hintergrunddruck \SI{1,9e-9}{\torr}


 
\subsection{Einheiten in Tabellen}
Die Verwendung von Einheiten in Tabellen mit dem Paket \emph{siunitx} ist in Tabelle \ref{Tab:BspS} gezeigt.
\begin{table}
\centering
\begin{tabular}{lSSSSS}
\hline\hline
 & {He} &  {Ne} & {Ar} & {Kr} & {Xe}\\
\hline
$\Gamma_{ch}$ [\SI{e16}{\metre\tothe{\num{-2.15}}\kelvin\tothe{\num{-1.2875}}}] & 36,189 & 31,07 & 3,495 & 1,9531 & 231,036\\
$K_{ch}$ [\si{\K\tothe{\num{2.2875}}\per\milli\bar\micro\metre\tothe{\num{0.85}}}] & 34,865 & 18,5 & 16,46 & 2980,982 & 5,554\\
\hline\hline
\end{tabular}
\caption[Beispiel für die \emph{siunitx}-Klasse \enquote{S} in Tabellen]{Mit der Klasse \enquote{S} des \emph{siunitx}-Pakets werden Zahlen in Tabellen am Dezimaltrennzeichen ausgerichtet. Normaler Text muss dann in geschweiften Klammern gesetzt werden.}
\label{Tab:BspS}
\end{table}


\section{Blindtext}
\end{comment}
