\chapter{Programmüberblick}
Überblick über die erstellten Matlab-Funktionen. Der vollständige Programmcode ist unter XXX abrufbar.
 
%\subsection*{Hilfsfunktionen}
%\begin{description}[style=nextline]
%	\item [\textit{[data]=\textsc{ft2}(data)}]
%		GPU-optimierte Version von fftshift(fft2(fftshift(data))) für grade N
%	\item [\textit{[data]=\textsc{ift2}(data)}]
%		GPU-optimierte Version von fftshift(ifft2(fftshift(data))) für grade N
%\end{description}
\subsection*{Erzeugung von Objekten}
\subsection*{Simulation}
\begin{description}[style=nextline]
	\item [\textit{[theta,Intensity,S1,S2]=\textsc{mie}(lambda,radius,beta,delta,steps)}]
		Intensität in Mie Streuung unpolarizierten Lichtes an Sphäre mit Radius 'radius' (in nm)
		und Brechzahl n=1-delta+ibeta bei Wellenlänge lambda (in nm), ausgewertet in
		'steps' linearen Schritten des Winkels theta)
\end{description}
\begin{description}[style=nextline]
	\item [\textit{[out]=\textsc{msft}(wavelength,objects,N,dx,gpu)}]
\end{description}
\begin{description}[style=nextline]
	\item [\textit{[out]=\textsc{multislice}(wavelength,objects,N,dx,gpu)}]
\end{description}
\begin{description}[style=nextline]
	\item [\textit{[out]=\textsc{thibault}(wavelength,objects,N,dx,gpu)}]
\end{description}



\subsection*{Rekonstruktion}
\begin{comment}


Hilfmittel
	ft2
	ift2
	maskfilter
	pad2size
Simulation
	Erzeugun von Objekten
		
	-msft
	-multislice
	-thibault

Rekonstruktion
	-wiener
	-reconstruct
	SW
	holo\_support
	ERiter
	RAARiter
	HIOiter
	\end{comment}